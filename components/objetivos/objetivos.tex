\section{Objetivo general}

  Realizar un análisis estadístico-descriptivo que permita evaluar el impacto del uso de herramientas de inteligencia artificial en el rendimiento académico de los alumnos de la Universidad Tecnológica del Perú (UTP) durante el periodo académico 2024 I - II, identificando patrones, tendencias y relaciones significativas que contribuyan a una mejor comprensión de cómo estas tecnologías afectan el proceso de aprendizaje.
  
  
  \subsection{Objetivos específicos}
  
  \begin{enumerate}
    \item \textbf{Identificar y analizar} las herramientas de inteligencia artificial más utilizadas por los alumnos de la UTP y caracterizar su funcionalidad, así como su propósito en el proceso de aprendizaje.
  
    \item \textbf{Recopilar y clasificar} datos de rendimiento académico de los alumnos antes y después de la implementación de herramientas de inteligencia artificial, con el fin de facilitar la comparación y evaluación de su efectividad.
  
    \item \textbf{Calcular y comparar} las medidas de tendencia central (media, mediana y moda) de las calificaciones de los estudiantes que utilizan herramientas de inteligencia artificial versus aquellos que no las utilizan, para determinar la influencia de estas tecnologías en el rendimiento académico.
  
    \item \textbf{Evaluar la percepción} de los alumnos sobre el uso de herramientas de inteligencia artificial en su aprendizaje mediante encuestas, identificando las ventajas y desventajas que ellos perciben.
  
    \item \textbf{Analizar las diferencias} en el rendimiento académico entre diferentes grupos de estudiantes (por ejemplo, por carrera o nivel académico) que utilizan herramientas de inteligencia artificial, buscando patrones que puedan indicar un efecto significativo de estas tecnologías.
  
    \item \textbf{Investigar la relación} entre la frecuencia de uso de herramientas de inteligencia artificial y el rendimiento académico de los estudiantes, para entender mejor cómo la integración de estas tecnologías puede optimizar el aprendizaje.
  \end{enumerate}