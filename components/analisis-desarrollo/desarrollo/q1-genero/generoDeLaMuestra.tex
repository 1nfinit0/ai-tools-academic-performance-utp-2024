\subsection{Género de la muestra:}
  
  \textbf{Tabla de frecuencias:}
  
  El siguiente cuadro muestra la distribución de género de los participantes en la muestra, dividiéndose en masculino y femenino, así mismo se muestra la frecuencua absoluta, acumulada, relativa porcentual y relativa porcentual acumulada.

  \begin{multicols}{2}
    \begin{enumerate}
      \item \begin{center}
        \textbf{Marca de Clase: $x_i$}
        \hrulefill
        \begin{equation*}
            x_i = \frac{L_i + L_s}{2}
        \end{equation*}
    \end{center}
    \vspace{-0.5cm}
    Donde:
    \begin{itemize}
        \item $L_i$: Límite inferior de la clase.
        \item $L_s$: Límite superior de la clase.
    \end{itemize}
    Para este caso, la marca de clase para valores cualitativos es el valor mismo por lo que $x_1 = Masculino$ y $x_2 = Femenino$.

    \item \begin{center}
        \textbf{Frecuencia Absoluta: $f_i$}
        \hrulefill
        \begin{equation*}
          f_i = \sum_{i=1}^{n} x_i
        \end{equation*}
    \end{center}
    \vspace{-0.7cm}
    Donde:
    \begin{itemize}
        \item $x_i$: Marca de clase.
        \item $f_i$: Frecuencia absoluta.
    \end{itemize}
    Para este caso, $f_1 = 51$ y $f_2 = 26$.

    \item \begin{center}
      \textbf{Frecuencia Acumulada: $F_i$}
      \hrulefill
      \begin{equation*}
        F_i = \sum_{i=1}^{n} f_i
      \end{equation*}
    \end{center}
    \vspace{-0.5cm}
    Donde:
    \begin{itemize}
        \item $f_i$: Frecuencia absoluta.
        \item $F_i$: Frecuencia acumulada.
    \end{itemize}

    \item \begin{center}
      \textbf{Frecuencia Relativa \%: $h_i\%$}
      \hrulefill
      \begin{equation*}
        h_i = \frac{f_i}{n} \times 100
      \end{equation*}
    \end{center}
    \vspace{-1cm}
    Donde:
    \begin{itemize}
        \item $f_i$: Frecuencia absoluta.
        \item $h_i$: Frecuencia relativa porcentual.
    \end{itemize}

    \item \begin{center}
      \textbf{Frecuencia Acumulada \%: $H_i\%$}
      \hrulefill
      \begin{equation*}
        H_i = \sum_{i=1}^{n} h_i
      \end{equation*}
    \end{center}
    \vspace{-1cm}
    Donde:
    \begin{itemize}
        \item $h_i$: Frecuencia relativa porcentual.
        \item $H_i$: Frecuencia acumulada porcentual.
    \end{itemize}
    \end{enumerate}
  \end{multicols}
  \vspace{-0.5cm}
  Finalmente se muestra la tabla de frecuencias:

  \renewcommand{\arraystretch}{1.5} % Ajusta la altura de las filas
\begin{table}[ht]
    \centering
    \begin{tabular}{l @{\hskip 0.5cm} c @{\hskip 0.5cm} c @{\hskip 0.5cm} c @{\hskip 0.5cm} c}
      \hline
      \textbf{Género} & \textbf{$f_i$} & \textbf{$F_i$} & \textbf{$h_i$ (\%)} & \textbf{$H_i$ (\%)} \\ \hline
      Masculino       & 51             & 51             & 66.23               & 66.23               \\ \hline
      Femenino        & 26             & 77             & 33.77               & 100.00              \\ \hline
      Total           & 77             &                & 100.00              &                     \\
    \end{tabular}
    \caption{Distribución por género con frecuencias absolutas, acumuladas y relativas.}
    \label{tab:genero-frecuencias}
\end{table}

\textbf{Interpretación de resultados:}

La tabla de frecuencias muestra que el 66.23\% de los participantes en la muestra son de género masculino, mientras que el 33.77\% son de género femenino. La distribución de género en la muestra refleja una mayor representación de estudiantes masculinos en comparación con las estudiantes femeninas.