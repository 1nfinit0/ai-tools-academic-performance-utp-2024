\subsection{Impacto de la I.A. en la capacidad de la investigación independiente}
\textbf{Tabla de Frecuencias:}
\begin{table}[H]
	\centering
	\renewcommand{\arraystretch}{1.2}
	\begin{tabular}{l c c c}
		\hline
		{Respuesta} & {\(f_i\)} & \textit{Fi} & \textit{hi}(\%)\\
		\hline
		Ha aumentado ligeramente         & 20 & 20 & 25.97\%\\
		Ha aumentado significativamente  & 3  & 23 & 3.90\%\\
		Ha disminuido ligeramente        & 22 & 45 & 28.58\%\\
		Ha disminuido significativamente & 5  & 50 & 6.49\%\\
		No ha tenido impacto             & 27 & 77 & 35.06\%\\
		\hline
		Total                            & 77 &    & 100\%\\
		\hline
	\end{tabular}
	\caption{Distribucion de la percepción del impacto}
	\label{tabla:investigacionIndependiente}
\end{table}

\textbf{Moda:} La respuesta más común es "No ha tenido impacto", con un total de 27 respuestas (35.06\%), lo que refleja que esta opción fue seleccionada por una parte considerable de la muestra.

\textbf{Mediana:} La mediana se sitúa entre las posiciones 38 y 39, por lo que pertenece a "No ha tenido impacto", demostrando que una proporción importante de los alumnos de la UTP no percibe un cambio significativo en su capacidad para realizar investigaciones independientes como resultado del uso de ChatGPT.

\textbf{Interpretación de los resultados:}\\
El 35.06\% de los estudiantes considera que el uso de ChatGPT no ha influido en su capacidad para investigar de manera autónoma. Un 28.58\% opina que ha habido una leve disminución en esta habilidad, mientras que un porcentaje similar (25.97\%) menciona que ha experimentado un incremento moderado en su capacidad investigativa. Solo un 6.49\% indica una disminución significativa, y un pequeño porcentaje (3.90\%) señala una mejora considerable en su capacidad de investigación.
