\subsection{Sobre la calidad de respuesta de la I.A.}

Los estudiantes universitarios fueron consultados acerca de la calidad de respuesta de la inteligencia artificial. A continuación, se presenta la distribución de las respuestas obtenidas:

% Respuestas fi Fi hi (%) Hi (%)
% Vacío 1 1 1.30 % 1.30 %
% Muy baja 1 2 1.30 % 2.60 %
% Baja 3 5 3.90 % 6.50 %
% Regular 45 50 58.44 % 64.94 %
% Alta 23 73 29.87 % 94.81 %
% Muy alta 4 77 5.19 % 100 %
% Total 77 100 %

\begin{table}[H]
  \centering
  \renewcommand{\arraystretch}{1.2}
  \begin{tabular}{l c c c c}
    \hline
    {Respuestas} & {\(f_i\)} & \textit{Fi} & \textit{hi}(\%) & \textit{Hi}(\%)\\
    \hline
    Vacío    & 1  & 1  & 1.30\%  & 1.30\%\\
    Muy baja & 1  & 2  & 1.30\%  & 2.60\%\\
    Baja     & 3  & 5  & 3.90\%  & 6.50\%\\
    Regular  & 45 & 50 & 58.44\% & 64.94\%\\
    Alta     & 23 & 73 & 29.87\% & 94.81\%\\
    Muy alta & 4  & 77 & 5.19\%  & 100\%\\
    \hline
    Total    & 77 &    & 100\%   & \\
    \hline
  \end{tabular}
  \caption{Distribución de respuestas sobre la calidad de respuesta de la I.A.}
  \label{tabla:calidadRespuesta}
\end{table}

\textbf{Moda $M_o$:} De un total de 77 respuestas obtenidas como muestra, la opción "Regular" tiene la 
frecuencia más alta, representando el 58.44\%, lo que equivale a 45 respuestas.

\textbf{Mediana $M_e$:} La mediana se encuentra en la posición 39, la cual corresponde a la opción "Regular", lo que indica que la mediana de las evaluaciones de los estudiantes sobre la precisión y calidad de las respuestas de la I.A. se encuentra en la categoría de calificación regular.

\textbf{Interpretación de resultados:}

La mayoría de los estudiantes encuestados de la UTP durante el periodo 2024 I-II evalúan la precisión y calidad de las respuestas proporcionadas por herramientas de inteligencia artificial (I.A.) como "Regular", con un total de 45 respuestas a favor (58.44\%), indicando de esta manera que más de la mitad de los encuestados consideran que estas herramientas cumplen un nivel aceptable. La segunda categoría más seleccionada es "Alta", con 23 respuestas (29.87\%), seguida de "Muy alta", con 4 respuestas (5.19\%), estos alumnos cuentan con una percepción positiva de las herramientas valorando su capacidad para proporcionar respuestas útiles y confiables. Sin embargo, también contamos con percepciones negativas en las categorías "Muy baja", "Baja" y "Vacío" acumulan solo 5 respuestas en total (6.50\%). Esto sugiere que una minoría de los encuestados considera que estas herramientas ofrecen una calidad insuficiente.
