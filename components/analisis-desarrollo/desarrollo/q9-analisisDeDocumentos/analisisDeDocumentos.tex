\subsection{Propósito del uso de la función de análisis de documentos}
\textbf{Tabla de Frecuencias:}
\begin{table}[H]
	\centering
	\renewcommand{\arraystretch}{1.2}
	\begin{tabular}{l c c c}
		\hline
		{Respuesta} & {\(f_i\)} & \textit{Fi} & \textit{hi}(\%)\\
		\hline
		Para aclarar conceptos confusos        & 39 & 39 & 50.65\%\\
		Para generar material de estudio adicional & 11 & 50 & 14.28\%\\
		Para preparar exámenes                 & 2  & 52 & 2.60\%\\
		Para profundizar en temas específicos  & 16 & 68 & 20.78\%\\
		Para resolver ejercicios               & 9  & 77 & 11.69\%\\
		\hline
		Total                                  & 77 &    & 100\%\\
		\hline
	\end{tabular}
	\caption{Distribución de las respuestas}
	\label{tabla:analisisDeDocumentos}
\end{table}

\textbf{Moda:} La opción más frecuente es "Para aclarar conceptos confusos", con 39 respuestas, lo que representa el 50.65\% de los alumnos encuestados de la UTP.

\textbf{Mediana:} La mediana se sitúa entre las posiciones 38 y 39, que también encontramos en la categoría "Para aclarar conceptos confusos", indica que este es el uso principal para la mayoría de la muestra.

\textbf{Interpretación de los resultados:}\\
La mayoría de los alumnos (50.65\%) utiliza la función de análisis de documentos de ChatGPT para resolver dudas sobre conceptos difíciles. Un 20.78\% lo hace para profundizar en ciertos temas, mientras que un 14.28\% lo emplea para generar material adicional de estudio. Solo una fracción más reducida (menor al 12\%) lo utiliza para preparar exámenes (2.60\%) o resolver ejercicios (11.69%).
