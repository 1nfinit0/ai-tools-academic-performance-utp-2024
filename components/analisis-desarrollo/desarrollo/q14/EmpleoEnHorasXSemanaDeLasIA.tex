\subsection{Empleo en horas a la semana de las I.A.}
\noindent \textbf{Tabla de Frecuencias:} \\

\begin{table}[h!]
	\centering
	\renewcommand{\arraystretch}{1.5} 
	\begin{tabular}{l c c c }
		\hline
		{Rango de horas} & {\(f_i\)} & \textit{\(h_i\)}(\%) & \textit{\(x_i\)}\\
		\hline
		Menos de una hora	& 42 & \(55.26\%\) & \(0.5\)	\\
		1 - 5 horas			& 26 & \(34.21\%\) & \(3\)		\\
		6 - 10 horas 		& 6  & \(7.89\%\)  & \(8\)		\\
		11 - 20 horas 		& 2  & \(2.63\%\)  & \(15.5\)	\\
		Más de 20 horas		& 0  & \(0\%\)     & \(23\)		\\
		\hline
		Total				& 76 & \(100\%\) \\
		\hline
	\end{tabular}
	\caption{Empleo de las inteligencias artificiales en horas a la semana}
	\label{tabla:EmpleoEnHoras}
\end{table}

\noindent \textbf{Media(\(\bar{x}\)):}
\begin{equation*}
	\bar{x} = \frac{(0.5 \times 42) + (3 \times 26) + (8 \times 6) + (15.5 \times 2) + (23 \times 0)}{76} \approx 2.34
\end{equation*}
La media semanal del uso de las herramientas de inteligencia artificial es de 2.34 horas, este bajo promedio sugiere que, en general, los encuestados no están profundamente ligados al uso de estas herramientas \\ \\

\noindent\textbf{Varianza:} \\
Se usará varianza muestral.
\begin{equation*}
	s^2 = \frac{(0.5 - 2.34)^2 + (3 - 2.34)^2 + (8 - 2.34)^2 + (15.5 - 2.34)^2 + (23 - 2.34)^2}{75} \approx 8.48
\end{equation*} \\

\begin{minipage}[t]{0.5\textwidth}
	\noindent\textbf{Desviación Estandar:}
	\begin{equation*}
		s = \sqrt{8.48} \approx 2.91
	\end{equation*}
	
\end{minipage}%
\hfill
\begin{minipage}[t]{0.5\textwidth}
	\noindent\textbf{Coeficiente de Variación:}
	\begin{equation*}
		\frac{2.91}{2.34} \times 100 \approx 124.36
	\end{equation*}  
\end{minipage} \\ \\

\noindent Las medidas de disperción sugieren que el uso de las herramientas de IA es generalmente bajo entre los encuestados, con una sustancial variabilidad causada por algunos pocos que la usan por extensos periodos de tiempo.