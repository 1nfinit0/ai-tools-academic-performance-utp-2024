\subsection{Percepción del Impacto Negativo de la I.A. en el aprendizaje}
\textbf{Tabla de Frecuencias:}

\begin{table}[h!]
	\centering
	\renewcommand{\arraystretch}{1.5}
	\begin{tabular}{l c c }
		\hline
		Respuestas & \(f_i\) & \(h_i\) \\
		\hline
		Nada & 8 & \(10.53\%\) \\
		Poco & 49 & \(64.47\%\) \\
		De manera significativa & 15 & \(19.74\%\) \\
		Mucho & 4 & \(5.26\%\) \\
		Demasiado & 0 & \(0\%\) \\
		\hline
		Total & 76 & \(100\%\) \\
		\hline
	\end{tabular}
	\caption{Percepción del Impacto Negativo de la I.A. en el aprendizaje}
	\label{tabla:percepciónNegativaEnElAprendizaje}
\end{table}

\textbf{Moda:} En este caso, es "Poco" puesto que es la respuesta más seleccionada \\ \\
\textbf{Mediana:} Al tratarse de datos ordinales, la mediana es el punto medio, osea los datos en la posiciones 38 y 39  que tambien pertecen a la respuesta "Poco" \\ \\
\textbf{Interpretación de resultados:}

La mayoría de los encuestados considera que el uso de las inteligencias artificiales afecta poco a su aprendizaje