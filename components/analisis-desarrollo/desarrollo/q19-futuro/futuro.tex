\subsection{Sobre el futuro de la educación universitarial con la I.A.}

Con respecto a la pregunta \textbf{¿Cómo crees que las herramientas de IA afectarán el futuro de la educación universitaria?}, se obtuvieron las siguientes respuestas:

%Tendrán un impacto positivo - 26 - 33.8%
%Tendrán un impacto negativo - 22 - 28.6%
%No cambiarán mucho la situación actual - 15 - 19.5%
%o estoy seguro - 14 - 18.2%

\begin{table}[H]
  \centering
  \renewcommand{\arraystretch}{1.5}
  \begin{tabular}{l c c }
    \hline
    Respuestas & \(f_i\) & \(h_i\) \\
    \hline
    Tendrán un impacto positivo & 26 & \(33.8\%\) \\
    Tendrán un impacto negativo & 22 & \(28.6\%\) \\
    No cambiarán mucho la situación actual & 15 & \(19.5\%\) \\
    No estoy seguro & 14 & \(18.2\%\) \\
    \hline
    Total & 77 & \(100\%\) \\
  \end{tabular}
  \caption{Percepción sobre el futuro de la educación universitaria con la I.A.}
  \label{tabla:futuroIA}
\end{table}

\textbf{Experimento aleatorio \(\epsilon\) :} La pregunta planteada a los estudiantes universitarios.

\textbf{Espacio muestral \(\Omega\):} El espacio muestral está compuesto por todos los posibles resultados de la pregunta, es decir, las respuestas de los estudiantes universitarios.

\begin{itemize}
  \item Tendrán un impacto positivo
  \item Tendrán un impacto negativo
  \item No cambiarán mucho la situación actual
  \item No estoy seguro
\end{itemize}

\textbf{Cardinal del espacio muestral $|\Omega|$ :} Representa la cantidad de posibles resultados de la pregunta. Para este caso, \(|\Omega| = 4\).

\textbf{Por la definición clásica de probabilidad,} se puede calcular la probabilidad de que un estudiante universitario considere que las herramientas de IA tendrán un impacto positivo en la educación universitaria. Para ello, se utiliza la fórmula de la probabilidad clásica.

\begin{equation*}
  P(A) = \dfrac{n(A)}{n(\Omega)} = \dfrac{\text{Número de casos favorables}}{\text{Número de casos posibles}}
\end{equation*}

\textbf{Cálculo de la probabilidad:}

\begin{itemize}
  \item $P(\text{Tendrán un impacto positivo}) = \dfrac{26}{77} \approx 0.3377 \approx 33.77\%$
  \item $P(\text{Tendrán un impacto negativo}) = \dfrac{22}{77} \approx 0.2857 \approx 28.57\%$
  \item $P(\text{No cambiarán mucho la situación actual}) = \dfrac{15}{77} \approx 0.1948 \approx 19.48\%$
  \item $P(\text{No estoy seguro}) = \dfrac{14}{77} \approx 0.1818 \approx 18.18\%$
\end{itemize}

Por lo tanto, la probabilidad de que un estudiante universitario considere que las herramientas de IA tendrán un impacto positivo en la educación universitaria es de aproximadamente 33.77\%. Esto sugiere que la mayoría de los encuestados considera que las herramientas de IA tendrán un impacto positivo en la educación universitaria.